\documentclass[12pt,letterpaper]{article}
\usepackage{mla}
\usepackage{ifpdf}
\usepackage{setspace}
\usepackage[utf8]{inputenc}
\usepackage[english]{babel}

\doublespacing
\begin{document}
\begin{mla}{Jonah}{Eadie}{Ms. Smith}{English 2 HP, Period 2}{2 February 2015}
{On William Golding's Use of Jack's Varying Position of Power in The Lord of
the Flies}

It’s often been said that those who find themselves in positions of power are 
those who are least fit to yield it. William Golding derives much of Lord of 
the Flies’s meaning from this central idea of unfit and corrupt leadership, 
with much of the novel’s drama being found in the power struggle between the 
piece’s central characters, Ralph and Jack. Jack’s attempts to usurp power 
away from Ralph provide for the novel’s most biting social commentary: Jack, 
being one who has always found himself in positions of leadership, naturally 
expects that he will continue this trend and reign as the ruler of the 
castaway boys; and as the story progresses, Jack manages, through ruthless and 
base dealings, to shift the balance of power away from Ralph and towards 
himself. This overarching narrative of unjust power acts as commentary on the 
world in which Golding lived.

In the beginning of the novel, Jack naturally sees himself as the most fit 
ruler among the boys given his previous position of leadership in the choir, 
this being indicative of the “old-guard” power tre that mirrors the power 
struggles one finds in modern political life. When one is first introduced to 
Jack, one is given the image of a “natural leader”; he is found leading the 
choir boys, his most vehement supporters, to Ralph and Piggy’s tribunal. When 
deciding upon leadership, Jack is overtly upset by the group’s decision to 
cast Ralph as their leader, and just barely manages to contain his frustration 
over the decision. This inclusion of Ralph’s anger over the boys’ vote by 
Golding neatly sets the tone of the two’s power struggle found throughout the 
piece as its central drama, and gives the reader insight into Jack’s 
privileged and self-entitled placement of himself within his world. In effect, 
Jack acts as a symbol for the “old-guard” within the power structures and 
political system of the modern world. Jack, fully expecting power and respect 
simply because of his past experiences with power, represents aristocracy. The 
power he gains is not derived from merit or public opinion, but rather from 
past history of power and his own cunning. This directly plays into the 
author’s intended message of rule by the unjust.

Jack, through the course of the novel, uses his cunning and the inherited 
power granted to him through the choirboys to unseat Ralph, the elected leader 
of the boys, and places himself at the top of the islands’ sociopolitical 
food-chain, providing an analogue to modern political dealings one finds in 
the politics of today. Additionally, while Jack initially acts as a friend and 
ally to Ralph, the reader soon finds that this friendship ends when Jack 
realizes that Ralph will not be as easy to manipulate and to control as he 
finds the other boys. Thusly, Ralph stands his ground on the issue of the 
signal fire, a move that was not anticipated by Jack. This mirrors the 
political bullying one finds in modern political systems, with unjust actors, 
through immoral and amoral dealings, vying for positions of power that they 
otherwise would not be able to obtain.

It can be seen, then, that Golding used the character of Jack as an analogue 
for corrupt political actors.


\end{mla}
\end{document}
